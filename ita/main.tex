\documentclass[11pt,a4paper]{article}

% Pacchetti
\usepackage[utf8]{inputenc}
\usepackage[italian]{babel}
\usepackage[margin=2.5cm]{geometry}
\usepackage{hyperref}
\usepackage{listings}
\usepackage{xcolor}
\usepackage{graphicx}
\usepackage{enumitem}
\usepackage{tcolorbox}
\usepackage{fontawesome5}
\usepackage{booktabs}
\usepackage{fancyhdr}
\setlength{\headheight}{14pt}

% Configurazione hyperref
\hypersetup{
    colorlinks=true,
    linkcolor=blue,
    filecolor=magenta,      
    urlcolor=cyan,
    pdftitle={Guida Colab},
    pdfpagemode=FullScreen,
}

% Configurazione listings per Python
\lstdefinestyle{python}{
    language=Python,
    basicstyle=\ttfamily\small,
    keywordstyle=\color{blue}\bfseries,
    commentstyle=\color{gray}\itshape,
    stringstyle=\color{red},
    showstringspaces=false,
    breaklines=true,
    frame=single,
    numbers=left,
    numberstyle=\tiny\color{gray},
    backgroundcolor=\color{gray!10},
    tabsize=4
}

\lstset{style=python}

% Box personalizzati
\newtcolorbox{warningbox}{
    colback=orange!5!white,
    colframe=orange!75!black,
    title=\faExclamationTriangle\ Attenzione
}

\newtcolorbox{tipbox}{
    colback=blue!5!white,
    colframe=blue!75!black,
    title=\faLightbulb\ Suggerimento
}

\newtcolorbox{notebox}{
    colback=green!5!white,
    colframe=green!75!black,
    title=\faInfoCircle\ Nota
}

% Header e footer
\pagestyle{fancy}
\fancyhf{}
\rhead{\textit{Guida Colab}}
\rfoot{Pagina \thepage}

% Titolo
\title{\textbf{Guida per MulticoreTemplate notebook}\\\large}
\author{[Axel Rubini]}
\date{\today}

\begin{document}

\maketitle
\tableofcontents
\newpage

\section{Descrizione}
Questa guida fornisce istruzioni dettagliate per l' uso del notebook \texttt{MulticoreTemplate} in Google Colab.
I notebook sono gia' presenti nella repository dove e' presente questa guida.
\section{Prerequisiti}
\begin{itemize}
	\item Account Google (per accedere a Google Colab)
	\item GitHub (opzionale)
	\item Accesso a Google Drive
\end{itemize}

\section{Setup Iniziale}

\subsection{Apertura del Notebook}
\begin{enumerate}
	\item Vai su \url{https://colab.research.google.com/}
	\item Caricare il notebook dal computer:
	      \begin{itemize}
		      \item Clicca su \texttt{File > Carica notebook}
		      \item Seleziona la scheda \texttt{Carica} e scegli il file \texttt{MulticoreTemplate.ipynb} dal tuo computer
	      \end{itemize}

\end{enumerate}

\subsection{Configurazione Runtime}
\begin{enumerate}
	\item Vai su \texttt{Runtime > Cambia tipo di runtime}
	\item Seleziona le seguenti impostazioni:
	      \begin{itemize}
		      \item \textbf{Tipo di runtime}: Python 3
		      \item \textbf{Acceleratore hardware}: [GPU - T4]
	      \end{itemize}
\end{enumerate}

\subsection{Connessione al Runtime}
\begin{itemize}
	\item Clicca su \texttt{Connetti} in alto a destra
	      \includegraphics[width=0.3\textwidth]{../resources/ConnectionRunTime.png}
	\item Attendi che il runtime sia pronto (icona verde con checkmark)
\end{itemize}

\section{Struttura del Notebook}

\subsection{Sezione 1: [Configurazione Iniziale e Esercizi Futuri]}
\textbf{Scopo}: [In questa sezione del notebook sono presenti gli script che creano i repository per gli esercizi, ovvero verra' creata una main directory che conterra' le sottodirecotries per l' esercizio attuale e per i futuri esercizi.]

\textbf{Celle da eseguire}:
\begin{itemize}
	\item \texttt{[1]} - [La prima cella server per creare la main directory se non e' gia' presente. ed a creare le repository git per l' esercizio attuale]
	\item \texttt{[2]} - [In questa cella e' presente uno script che permette di visualizzare le caratteristiche della GPU.]
	\item \texttt{[3 (opzionale)]} - {[Questa cella configura git e crea una repository sul tuo account GitHub. L' accesso a GitHub deve essere eseguito tramite token personale, da inserire ogni volta che viene fatta eseguire la cella per la prima volta nel runtime e rimane valido fino a che il runtime non viene riavviato.]}
\end{itemize}

\begin{notebox}
	Se volete usare GitHub consiglio di creare un token e di salvarvelo dentro l' apposito gestore di chiavi di colab \\
	\includegraphics[width=0.3\textwidth]{../resources/CoLabKey.png}
\end{notebox}

\subsection{Sezione 2: [Caricare il propio codice]}
\textbf{Scopo}: [Caricare il proprio file.cu all' interno della directory dell' esercizio attuale.]

\textbf{Celle da eseguire}:
\begin{itemize}
	\item \texttt{[4]} - [In questa cella e' presente uno script che permette di caricare il proprio file.cu all' interno della direcotory dell' esercizio attuale, oppure utilizzare gia' i file esistenti all' interno del progetto.]
\end{itemize}

\begin{notebox}
	Se si vuole caricare un file che ha lo stesso nome di uno attuale e' consigliato eliminare il vecchio file a mano attraverso il terminale di lavoro o con il side tree presente su CoLab.
\end{notebox}
\subsection{Sezione 3: [Eseguire il codice e visualiccare i risultati]}
\textbf{Scopo}: [Le celle successive servono per compilare,pulire, eseguire e visualizzare i risultati del proprio codice all' interno del notebook.]

\textbf{Celle da eseguire}:
\begin{itemize}
	\item \texttt{[5]} - [Compila il codice sorgente .cu i parametri pre configurati sono quelli per il run su GPU T4, se si vuole cambiare tipo di GPU o altri parametri e' possibile modificare la cella.]
	\item \texttt{[6]} - [Esegue il clean della directory di lavoro, eliminando i file di output generati dalle esecuzioni precedenti.]
	\item \texttt{[7]} - [Esegue il codice compilato.]
\end{itemize}

\subsection{Sezione 4: [Salvataggio dei risultati e caricamento su GitHub]}
\textbf{Scopo}: [Le celle successive servono per generare un po' di documentazione e sincronizzare la repository su GitHub la descrizione e' omessa per brevita'.]



\end{document}
